\chapter{Introduction}
\section{Background}
\section{Problem description}
\section{Working Instruments}
\subsection{ACT-R}
\subsection{OpenCV}
	OpenCV, \emph{Open Source Computer Vision}, is a library for computer vision. It was originally developed by Intel, and now it is supported by Willow Garage.
	It is a cross-platform library and it is released under a BSD license, thus it is free and open source. It was developed in C and C++ and then interfaces for Java and Python were added. OpenCV was designed for computational efficiency and with a strong focus on real-time applications. The version 2.4 has more than 2500 algorithms. The library has been used in many applications as, for example, mine inspection and robotics \cite{OpenCVWebPage}.
	\begin{wrapfigure}{l}{0.2\textwidth}
  		\begin{center}
  	  	\includegraphics[width=0.2\textwidth]{images/ch_01/opencv_logo.png}
  		\end{center}
  		\caption{OpenCV Logo}
	\end{wrapfigure}		
	
	\subsubsection*{History}
	The OpenCV Project started in 1999 as an Intel Reasearch initiative aimed to improve CPU intensive applications as a part of projects including real-time ray tracing and 3D display walls. The early goals of the project were developing optimized code for basic vision infrastructure, spreading this infrastructure to developers, and making it portable and available for free without forcing the developers to create their applications for free.
	The first alpha version was released to the public in 2000, followed by five beta versions between 2001 and 2005, which lead to version 1.0 in 2006. In 2008, the technology incubator Willow Garage begun supporting the project and, in the same year, version 1.1  was released. 
	On October 2009, OpenCV 2.0 was released. It included many improvements, such as a better C++ interface, more programming patterns, new functions and multicore optimized implementations of old ones. Willow Garage releases official versions of OpenCV every six months \cite{wiki:OCVHistory}.
	\subsubsection*{Main Features}
		\includegraphics[width=1\textwidth]{images/ch_01/opencv_overview.jpg}
	The figure above shows the main features of the library. First of all, OpenCV offers an easy way to manage image data types. It provides functions to load, copy, edit, convert and store images and a basic graphical user interface that lets the developers handle keyboard and mouse and display image and video. The library can work both on images and videos and lets manipulate images even with matrix and vector algebra routines, for example, eigenvalues and singular value decomposition. It supports the most common dynamic data structures and offers many different basic image processing functions: filtering, edge and corner detection, color conversion, sampling and interpolation, morphological operations, histograms and image pyramids. Beyond this, it integrates many functions for structural analysis of the image including connected components, contour processing, distance transform, template matching, Hough transform, polygonal approximation, line fitting, ellipse fitting and Delaunay triangulation as well as features for camera calibration like finding and tracking calibration patterns, calibration, fundamental matrix estimation, homography estimation and stereo correspondence. It also offers many functions for motion analysis like optical flow, motion segmentation and tracking and features for object recognition \cite{Agam2006}.
